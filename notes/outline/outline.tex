\documentclass{article}

\usepackage{mhchem}

\begin{document}

\begin{enumerate}
\item Preface
    \begin{enumerate}
    \item Quantum materials are really cool
    \item Correlated systems are really cool
    \item Probably the most exciting systems right now
    \item What are the major questions?
    \item Description of the book - dissemination of results, and instruction manual
    \end{enumerate}
\item Introduction to ultrafast optics in correlated electron systems
    \begin{enumerate}
    \item Interactions: the weak limit
        \begin{enumerate}
        \item Fermi liquid
        \item CDW / SDW
        \end{enumerate}
    \item The strong limit
        \begin{enumerate}
        \item Mott insulator
        \item Localized spins
        \item Cuprates
        \end{enumerate}
    \item Spectroscopy
        \begin{enumerate}
        \item Comparison of nonequil. vs equil. spectroscopy
        \item Excitation of coherent collective modes
            \begin{enumerate}
            \item In correlated materials
            \item In multiferroics
            \end{enumerate}
        \end{enumerate}
    \item Control
        \begin{enumerate}
        \item Dynamical phase transitions
        \item TDGL theory
        \item Incoherent versus incoherent control
            \begin{enumerate}
            \item In correlated systems
            \item In CDWs
            \item In magnets
            \end{enumerate}
        \end{enumerate}
    \end{enumerate}
%     \item Meaning of strong interactions
%         \begin{enumerate}
%         \item Weak interactions: Fermi liquid
%         \item Repulsive interactions: leads to SDW (helicoidal and sinusoidal), Mott insulator
%         \item Attractive interactions: leads to CDW, exciton insulator, SC
%         \item If there is good nesting, these phases happen for arbitrarily weak interactions.
%         \item Need strong interactions if not.
%         \end{enumerate}
%         \end{enumerate}
%     \item Important aspects of correlated systems for this thesis
%         \begin{enumerate}
%         \item Charge density wave (example: 1\textit{T}-\ce{TaS2})
%         \item Kondo effect and Magnetic order (example: \ce{CaMn2Bi2})
%         \item Multiferroics (example: \ce{CuBr2})
%         \end{enumerate}
\item SHG theory
    \begin{enumerate}
    \item Space groups and point groups
    \item Response tensors
        \begin{enumerate}
        \item Multiple contributions to SHG
        \end{enumerate}
    \item Symmetry of tensors - the free energy
        \begin{enumerate}
        \item SHG does not mean inversion broken
        \item Presence of absorption
        \end{enumerate}
    \item Bond model
    \item Phenomenological GL model of SHG
        \begin{enumerate}
        \item Free energy is p dot e
        \item Free energy for SHG can be written like ...
        \item $\chi_{ijk} = \chi_{ijkl} O_l$ (but fully general) is a valid expression for the free energy
        \item Free energy needs to be a real and totally symmetric scalar
        \item This gives us constraints on $\chi_{ijk}$.
        \item Time reversal affects SHG
        \item SHG also measures domains
        \end{enumerate}
    \item Quantum model
        \begin{enumerate}
        \item Wavelength dependence of SHG
        \end{enumerate}
    \end{enumerate}
\item SHG practical
    \begin{enumerate}
    \item Basic idea
        \begin{enumerate}
        \item Connection to last chapter - want to probe as many elements as possible
        \item We have control over the fields and the outgoing polarization
        \item Our choices: oblique incidence, large spot size, single color, PMT detector
        \item Schematic description of the setup
        \end{enumerate}
    \item Before you build the setup
        \begin{enumerate}
        \item Choice of oblique vs. normal incidence - which tensor elements do you want to probe?
        \item Scaling of SHG signal with volume, pulse width - microscopy or not? Domain size?
        \end{enumerate}
    \item Construction of the setup
        \begin{enumerate}
        \item Description of the setup that we built
        \item Automated polarization rotators
        \item Better to use hollow bore stepper motors
        \item Choice of detector
        \item Alignment
            \begin{enumerate}
            \item Aligning the circles
            \item Checking that the symmetry looks good
            \item Waveplates
            \end{enumerate}
        \end{enumerate}
    \item Considerations for time-resolved
        \begin{enumerate}
        \item Location of pump mirror
        \item Alignment of normal incidence
        \item Choice of pump wavelength (OPA)
        \item Polarization rotation (why do it?)
        \item Pump scatter
        \end{enumerate}
    \item Data analysis (static)
    \item Data analysis (time-resolved)
    \end{enumerate}
\item Appendix A: Failed experiments
\end{enumerate}

\end{document}
