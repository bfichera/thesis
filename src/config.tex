\usepackage{tabularx}
\usepackage{newunicodechar}
\newunicodechar{∗}{*}
\usepackage{multirow}
\usepackage{palatino}
\usepackage{pdflscape}
\usepackage{amsmath}
\usepackage{amssymb}
\usepackage{textgreek}
\usepackage{graphicx}
\usepackage{svg}
\usepackage[numbers,sort&compress]{natbib}
\usepackage{blindtext}
\usepackage{xspace}
\usepackage{microtype}
\usepackage{mathrsfs}
\usepackage{siunitx}
\usepackage{mhchem}
\usepackage{braket}
\usepackage{nicefrac}
\usepackage{bm}
\usepackage{glossaries-extra}
\glsdisablehyper
\usepackage[inline, shortlabels]{enumitem}
\usepackage[caption=false]{subfig}
\captionsetup[subfigure]{singlelinecheck=off}
\renewcommand\thesubfigure{(\alph{subfigure})}
\newcommand{\phantomsubfloat}[1]{
    {% apply caption setup only temporarily
        \captionsetup[subfigure]{labelformat=empty}
        \subfloat[][]{#1}
    }%
}
\newcommand{\caplabel}{(\alph*)}
\newcommand{\capref}{(\alph*)}
\usepackage{hyperref}
\hypersetup{
    colorlinks=true,
    allcolors=cyan,
}
\usepackage{cleveref}

\crefname{figure}{Fig.}{Figs.}
\Crefname{figure}{Fig.}{Figs.}
\crefrangelabelformat{figure}{#3#1#4--#5#2#6}
\crefname{subfigure}{Fig.}{Figs.}
\Crefname{subfigure}{Fig.}{Figs.}
\crefrangelabelformat{subfigure}{#3#1#4--#5\crefstripprefix{#1}{#2}#6}
\crefname{equation}{Eq.}{Eqs.}
\Crefname{equation}{Eq.}{Eqs.}
\crefrangelabelformat{equation}{#3#1#4--#5#2#6}
\creflabelformat{equation}{#2#1#3}

\crefmultiformat{subfigure}{\edef\crefstripprefixinfo{#1}figs.~#2#1#3}{ and~#2\crefstripprefix{\crefstripprefixinfo}{#1}#3}{, #2\crefstripprefix{\crefstripprefixinfo}{#1}#3}{, and~#2\crefstripprefix{\crefstripprefixinfo}{#1}#3}

\newignoredglossary{ignored}
\newcommand*\nestedglsentry[1]{%
  \protect\ifglsused{#1}{%
    \glsentryshort{#1}%
  }{% 
    \glsentrylong{#1}%
  }%  
}


\bibstyle{plainnat_noclutter}
\bibliographystyle{plainnat_noclutter}

% Lengths
\newlength{\linespace} \setlength{\linespace}{\baselineskip}
\newlength{\topfiddle} \setlength{\topfiddle}{2\linespace}

\setsecnumdepth{subsubsection}
\settocdepth{subsubsection}


\newcommand{\defendmonth}{May\xspace}
\newcommand{\defendday}{?\xspace}
\newcommand{\defendyear}{2024\xspace}
\newcommand{\defenddate}{\defendmonth \defendday, \defendyear}

\newcommand{\pretoctitle}[1]{{\Large\bfseries\clearpage\vspace{\topfiddle}#1\par\vspace{\topfiddle}}}
\newcommand{\acknowledgements}{\pretoctitle{Acknowledgements}}

% \newcommand{\etal}{\textit{et al.}\xspace}

% \newcommand{\onlinecref}[1]{Ref.~\citenum{#1}\xspace}


\setabbreviationstyle[ignored]{long-noshort}
\setabbreviationstyle[abbreviation]{long-short}
\newabbreviation[type=ignored]{lro}{LRO}{long-range order}
\newabbreviation[type=ignored]{op}{OP}{order parameter}
\newabbreviation{sc}{SC}{superconductivity}
\newabbreviation{shg}{SHG}{second harmonic generation}
\newabbreviation{afm}{AFM}{antiferromagnetic}
\newabbreviation{rashg}{RA-SHG}{rotational anisotropy \nestedglsentry{shg}}
\newabbreviation{trshg}{tr-SHG}{time-resolved \nestedglsentry{shg}}
\newabbreviation{cdw}{CDW}{charge density wave}
\newabbreviation{opa}{OPA}{optical parametric amplifier}
\newabbreviation{pmt}{PMT}{photomultiplier tube}
\newabbreviation{na}{NA}{numerical aperture}
\newabbreviation{emccd}{EM-CCD}{electron multiplying charge coupled device}
\newabbreviation{nir}{near-infrared}{NIR}
\newabbreviation[type=ignored]{dftr}{DFT}{discrete Fourier transform}
\newabbreviation{ft}{FT}{Fourier transform}
\newabbreviation{sbhm}{SBHM}{simplified bond hyperpolarizability model}
\newabbreviation{cif}{CIF}{crystallographic information file}
\newabbreviation{lm}{LM}{Levenberg-Marquardt}
\newabbreviation{ued}{UED}{ultrafast electron diffraction}
\newabbreviation{ic}{IC}{incommensurate}
\newabbreviation{nc}{NC}{nearly commensurate}
\newabbreviation{paw}{PAW}{projected augmented wave method}
\newabbreviation{vasp}{VASP}{Vienna Ab-initio Software Package}
\newabbreviation{gga}{GGA}{generalized gradient approximation}
\newabbreviation{pdos}{PDOS}{partial density of states}
\newabbreviation{mae}{MAE}{magnetocrystalline anisotropy energy}
\newabbreviation{cmr}{CMR}{colossal magnetoresistance}
\newabbreviation{xrd}{XRD}{X-ray diffraction}
\newabbreviation{ptmb}{PTMB}{photo-thermal modulated birefringence}
\newabbreviation{ife}{IFE}{inverse Faraday effect}
\newabbreviation{decp}{DECP}{displacive excitation of coherent phonons}
\newabbreviation{dfg}{DFG}{difference frequency generation}
\newabbreviation{isrs}{ISRS}{impulsive stimulated raman scattering}
\newabbreviation{gl}{GL}{Ginzburg-Landau}
\newabbreviation{icme}{ICME}{inverse Cotton-Mouton effect}
\newabbreviation{tdgl}{TDGL}{time-dependent Ginzburg-Landau}
\newabbreviation{ins}{INS}{inelastic neutron scattering}
\newabbreviation{lswt}{LSWT}{linear spin wave theory}
\newabbreviation{qcp}{QCP}{quantum critical point}
\newabbreviation{dft}{DFT}{density functional theory}

\newcommand{\suchthat}{\mathrm{s.t.}}
\newcommand{\tastwo}{1\textit{T}-\ce{TaS2}\xspace}
\newcommand{\cmb}{\ce{CaMn2Bi2}}
\newcommand{\op}{\mathscr{O}}
\newcommand{\PP}{$P_\mathrm{in}P_\mathrm{out}$\xspace}
\newcommand{\PS}{$P_\mathrm{in}S_\mathrm{out}$\xspace}
\newcommand{\SP}{$S_\mathrm{in}P_\mathrm{out}$\xspace}
\renewcommand{\SS}{$S_\mathrm{in}S_\mathrm{out}$\xspace}
\newcommand{\mathPP}{P_\mathrm{in}P_\mathrm{out}}
\newcommand{\mathPS}{P_\mathrm{in}S_\mathrm{out}}
\newcommand{\mathSP}{S_\mathrm{in}P_\mathrm{out}}
\newcommand{\mathSS}{S_\mathrm{in}S_\mathrm{out}}
\newcommand{\ft}{\hat}
\newcommand{\params}{\theta}
\newcommand{\apx}{{\sim}\xspace}
\newcommand{\threshold}{$\apx 200$ \si{\mu J \cdot cm^{-2}}\xspace}
\newcommand{\previousthreshold}{$\apx 10$ \si{mJ \cdot cm^{-2}}\xspace}
\newcommand{\chione}{\chi^\alpha}
\newcommand{\chitwo}{\chi^\beta}
\newcommand{\pt}{$\mathcal{PT}$\xspace}
\newcommand{\htpgmathmode}{\bar{3}m\xspace}
\newcommand{\htpg}{$\htpgmathmode$\xspace}
\newcommand{\insets}{Insets show the corresponding value of the \gls{afm} \gls{op} proposed in the text, relative to the hexagonal crystallographic motif.\xspace}
\newcommand{\lastrow}{The last row depicts the corresponding value of the \gls{afm} \gls{op} proposed in the text, relative to the hexagonal crystallographic motif.\xspace}
\newcommand{\degree}{^\circ?}
\newcommand{\supp}{Supplementary material}
\newcommand{\supcref}[1]{Supplementary material, \cref{#1}}
\newcommand{\xrd}{cmb-xrd}
\newcommand{\fulltempdep}{cmb-fulltempdep}
\newcommand{\fullequilibrium}{cmb-fullequilibrium}
\newcommand{\fullnonequilibrium}{cmb-fullnonequilibrium}
\newcommand{\fitting}{cmb-fitting}
\newcommand{\polarization}{cmb-polarization}
\newcommand{\CtoBshort}{cmb-CtoBshort}
\newcommand{\CtoBlong}{cmb-CtoBlong}
\newcommand{\errorbars}{cmb-errorbars}
\newcommand{\oned}{\num{1}D\xspace}
\newcommand{\neel}{N\'{e}el\xspace}
\newcommand{\higgs}{amplitude\xspace}
\newcommand{\ahiggs}{an amplitude\xspace}
\newcommand{\Higgs}{Amplitude\xspace}
\newcommand{\bigO}{\mathcal{O}}
\DeclareMathOperator{\atantwo}{atan2}

\newtheorem{theorem}{Theorem}[section]


\chapterstyle{brotherton}
\nouppercaseheads
\makeheadrule{headings}{\textwidth}{1pt}
\pagestyle{headings}
