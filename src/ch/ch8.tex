% In this thesis, we have broadly....
% We illustrated how SHG can be used....
% and have given a tutorial....
% We improved on this technique by....
% We demonstrated how SHG could measure....
% Taken together, these results....
% 
% At the end of this work, it is necessary to give some perspectives on where this field is going...
% we have not solved the correlated electron problem....more systematic studies are needed.....example, organic mott insulators
% more technology is needed....
% not sure if now is a good time, since there's still a lot of phenomenology to be discovered...
% nevertheless, at some point in the future, we will need systematic studies....
% difficult because of current publication climate....publish or perish....
% difficult to publish on manganites, etc....
% need a null or boring results journal / server

We began this thesis by discussing how strong interactions may invalidate the Fermi liquid model and result in novel quantum ground states which may not be understood within a single-particle picture.
The result is a situation where there is no general theory at present capable of predicting, \textit{a priori}, the ground state and low-energy excitations of an arbitrary correlated quantum material.
We have obviously not come any closer to such a theory over the course of this research---although, in truth, that was never really the goal.
Rather, if the goal was simply to enlarge the list of novel phenomena which can occur in quantum materials (acknowledging, of course, that this list is essential for solving the correlated electron problem), this research has, in some sense, succeeded.
Using a novel experimental technique (\gls{trshg}), we showed that antiferromagnets may be manipulated nonthermally by quenching the equilibrium order on ultrafast timescales, in a fashion which is similar in spirit to previous works on \gls{cdw} systems \citep{fausti_light-induced_2011,kogar_light-induced_2020}.
We also showed that spin-spiral multiferroic materials may host novel electromagnon excitations which are due not to the pseudo-Goldstone mode, but to the amplitude mode of the magnetic order.
Both of these discoveries were quite unexpected, and each of them \emph{required} measuring the full \gls{rashg} signal at each time delay---a technology that was developed (partially by us) only over the course of the last \num{10}--\num{15} years.
The story, then, is that a new technique was developed, and then that technique led us to new physics that we didn't previously know existed.

This story is quite common in condensed matter physics.
Consider, for example, the discovery of antiferromagnetism, which is no doubt more common in solids than ferromagnetism.
Nevertheless, ferromagnetism induces a net magnetic dipole moment, and the resulting macroscopic magnetic properties are thought to have been understood by humans as early as the fourth millenium B.C. \citep{magnetism_fundamentals}.
Antiferromagnets, in contrast, took until the \nth{20} century to be predicted \citep{neel_properties_1948} and experimentally discovered \citep{shull_neutron_1951}---the latter required the development of neutron scattering.
Thus, new techniques which are sensitive to different order parameters or excitations very often lead to new discoveries in condensed matter physics.

In light of this history, in my view there is no reason to think that the technology we have at present is sufficient to solve the correlated electron problem.
This is a perspective which is often voiced in the context of \ce{URu2Si2}, a heavy-fermion material with a phase transition at \qty{17.5}{K} to an ordered state whose order parameter and low-energy excitations remain unknown despite nearly thirty years of investigation \citep{palstra_superconducting_1985}.
It is thought \citep{alexandradinata_future_2022} that the puzzle has remained unsolved simply because we do not, at present, have an experimental probe which couples to the order in this compound.
Going forward, I thus predict that technique development will be the fundamental driving force for progress in strongly correlated electron physics.
A large part of this will involve \emph{improving} techniques that we already have (using, for example, artificial intelligence, or \nth{4}-generation synchrotron sources) to increase signal-to-noise, ease of data processing, etc.
A parallel effort will be to imagine totally different experimental probes (such as, for example, scattering of entangled particles\citep{shen_unveiling_2020}), although this is obviously more difficult.

There is some criticism to be made of the \emph{approach} we take to the correlated electron problem in the ``publish or perish'' era of modern science.
At the time of writing, a large portion of the research which is published in high-impact factor journals in this field is of the type pursued in this thesis---systematic research is largely \emph{avoided} in favor of surprising or unexpected results, which are, in the most pessimistic view, easier for journals to shape into a headline.
My feelings on this issue are somewhat mixed.
In the worst case, the current research climate deters researchers from pursuing systematic research, on the basis that it is more difficult to publish.
At the same time, the field is still somewhat in its infancy, and I do think that there are a large number of fundamental phenomena in quantum materials that simply have yet to be discovered.
Perhaps, then, the right research to be doing is indeed more exploratory right now than would otherwise feel comfortable.

In either case, I \emph{am} sure that there is \emph{absolutely no reason} to be publishing in for-profit journals!
Perhaps this would be understandable if the for-profit journals offered a monetary incentive to authors who elect to publish there.
But the reality is the opposite---in the publishing industry, the \emph{suppliers} pay for the goods that they provide, and then pay again to consume those goods.
This bizarre circumstance is possible only because of a valuation heuristic (which occurs at all levels of academic discourse---from faculty hiring decisions to grad lounge chit-chat) that ranks papers on the basis of which journal they are published in.
This results in a situation where the publishing industry itself holds all of the power in directing what scientists decide to study; thus the conventional wisdom that one should avoid manganites and cuprates since these are not of interest to editors of high-impact factor journals.
This is, in my opinion, the biggest impediment to progress right now in correlated electron physics.

One immediate solution would be the adoption of, for example, a ``journal of null or uninteresting results'',\footnote{See, for example, the Series of Unsurprising Results in Economics (SURE).} where authors could receive citations for research which is not flashy but may nevertheless be useful.
In the long term, however, the field must transition away from the for-profit publication industry, which may involve the difficult task of redefining our research valuation metholodgy.
To some extent I think this is already happening; many of the graduate students and postdocs I talk to share a similar point of view, and after all, those are the people that will eventually be making decisions about who gets tenure at academic institutions.

In any case, the idea that the correlated electron problem is important---something which I hope to have reinforced through the research conducted in this thesis---is certainly not something that will go away.
There is thus a bright future ahead for this field, and I am quite confident in the condensed matter community to help make that future a reality.
