Interactions between electrons in solids can be treated broadly in two different ways, depending on the strength of the interaction.
Let us start with the case where the interactions are weak compared to their kinetic energy (or the bandwidth $W$).
In this case, interactions may be treated as a perturbation
\begin{equation}
V = \sum_{\vec{p}\vec{p}'\vec{q},\sigma\sigma'}V(\vec{q})c^\dagger_{\vec{p}+\vec{q},\sigma}c^\dagger_{\vec{p}'-q,\sigma'}c_{\vec{p}',\sigma'}c_{\vec{p},\sigma}
\end{equation}
on top of the usual hamiltonian
\begin{equation}
H_0 = \epsilon_{\vec{p}} c^\dagger_{\vec{p},\sigma}c_{\vec{p},\sigma}
\end{equation}
for the noninteracting electrons.
The key insight is that the electrons are fermions, and thus obey the Pauli exclusion principle.
Consider, then, a single particle excited to an energy $\delta\epsilon$ above the Fermi surface, with $\delta\epsilon \ll \epsilon_F$; due to the Pauli exclusion principle, the volume of phase space available for scattering (at zero temperature) between this and other particles is confined to a shell around the Fermi surface with some thickness which goes to zero as $\delta\epsilon\rightarrow 0$.
Thus, by Fermi's golden rule, the lifetime of such particles approaches infinity as we get closer and closer to the Fermi surface; i.e., there are still well-defined quasiparticles there, even though we have an interacting Hamiltonian.

The ground state at low temperatures therefore just looks like a regular old metal, with a few important differences.
Mainly, it is not quite correct to call the quasiparticles electrons any more; really, they are collective excitations of the entire ensemble which nevertheless have a long quasiparticle lifetime.
These quasiparticles are like electrons, but with a renormalized mass
\begin{equation}
m^*=(1+\lambda)m
\end{equation}
for some $\lambda_m$ which depends on the nature (symmetry, energy scale, etc.) of the interaction.
Like the quasiparticle mass, many other important properties of the system are renormalized, such as the specific heat
\begin{equation}
c=\frac{m^*}{m}c_0,
\end{equation}
the magnetic susceptibility
\begin{equation}
\chi=\frac{m^*}{m}\frac{1}{1+\lambda_\chi}{\chi_0},
\end{equation}
or the compressibility
\begin{equation}
\kappa=\frac{m^*}{m}\frac{1}{1+\lambda_\kappa}{\kappa_0}.
\end{equation}
