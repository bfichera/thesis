Interactions between electrons in solids can be treated broadly in two different ways, depending on the strength of the interaction.
Let us start with the case where the interactions are weak compared to their kinetic energy.
In this case, interactions may be treated as a perturbation
\begin{equation}\label{eq:interactionhamiltonian}
V = \sum_{\vec{p}\vec{p}'\vec{q},\sigma\sigma'}V(\vec{q})c^\dagger_{\vec{p}+\vec{q},\sigma}c^\dagger_{\vec{p}'-q,\sigma'}c_{\vec{p}',\sigma'}c_{\vec{p},\sigma}
\end{equation}
on top of the usual hamiltonian
\begin{equation}\label{eq:kinetichamiltonian}
H_0 = \epsilon_{\vec{p}} c^\dagger_{\vec{p},\sigma}c_{\vec{p},\sigma}
\end{equation}
for the noninteracting electrons.
The key insight, due conceptually to \citet{landau} and later formalized by \citet{gell-mann}, is that, as long as the interactions are sufficiently weak, this problem can be adiabatically connected to a similar problem with non-interacting quasiparticles.
Since these quasiparticles are fermions, they obey the Pauli exclusion principle, and the phase space available for scattering low-energy excitations is thus goes to zero at small energies.
By Fermi's golden rule, the lifetime of such particles thus approaches infinity as we get closer and closer to the Fermi surface; i.e., there are still well-defined quasiparticles there, even though we started with an interacting Hamiltonian.
Such a system is hence referred as a ``Fermi liquid,'' to signify the fact that we have made only a slight departure from the nominal model of a Fermi gas where the particles are treated as noninteracting.

Fermi liquid theory is almost unreasonably successful in describing the ground state of a large number of correlated electron systems.
Truly \emph{most} metals can be very adequately described as a simple Fermi gas with renormalized effective mass, specific heat, etc.; while these modifications may be large (e.g. $m^*/m\approx\num{e3}$ in \ce{CeAl3}), they otherwise look like normal metals.
However, we can also discuss the circumstances in which the theory fails.
Clearly various electronic instabilities may gap out the Fermi surface, resulting in an insulator; this happens for arbitrarily weak interactions in quasi-\oned systems due to the Peierls mechanism, and may occur in higher dimensions due to, e.g., Fermi surface nesting.
Whether the interaction is attractive or repulsive determines whether the instability occurs in the charge or the spin channel.
In the other limit, where the kinetic energy hamiltonian \ref{eq:kinetichamiltonian} is treated as a perturbation to the interaction term \ref{eq:interactionhamiltonian}, the Fermi liquid state gives way to an (Mott) insulating state where the electrons become localized so as to minimize the coulomb repulsion $U$.

% Thus, instabilities in the Fermi liquid occur both in the weak and strongly interacting limits.



% If the interactions are too strong, however, the adiabaticity approximation fails, and the energies that we are interested in studying now involve excitations which 

% The ground state at low temperatures therefore just looks like a regular old metal, with a few important differences.
% Mainly, it is not quite correct to call the quasiparticles electrons any more; really, they are collective excitations of the entire ensemble which nevertheless have a long quasiparticle lifetime.
% These quasiparticles are like electrons, but with a renormalized mass
% \begin{equation}
% m^*=(1+\lambda)m
% \end{equation}
% for some $\lambda_m$ which depends on the nature (symmetry, energy scale, etc.) of the interaction.
% Like the quasiparticle mass, many other important properties of the system are renormalized, such as the specific heat
% \begin{equation}
% c=\frac{m^*}{m}c_0,
% \end{equation}
% the magnetic susceptibility
% \begin{equation}
% \chi=\frac{m^*}{m}\frac{1}{1+\lambda_\chi}{\chi_0},
% \end{equation}
% or the compressibility
% \begin{equation}
% \kappa=\frac{m^*}{m}\frac{1}{1+\lambda_\kappa}{\kappa_0}.
% \end{equation}


