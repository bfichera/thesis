\section{Space groups and point groups}

The utility of \gls{shg} in studying condensed matter systems is derived from the following simple statement, attributed to Franz Neumann\cite{neumann_vorlesungen_1885} and later Pierre Curie\cite{curie_sur_1894}:

\begin{theorem}[Neumann's principle]\label{thm:neumann}
Let $P_G$ be the symmetry group of a crystal structure and $P_H$ the symmetry group of some physical property of that crystal.
Then, $G$ is a subgroup of $H$.
\end{theorem}

There are a few things to digest here.
Let us start by understanding the meaning of the phrase ``symmetry group''.
For any given crystal, there exists some infinitely large set of operations $G$ under which the crystal structure is symmetric.
Each of these operations may be decomposed into two parts: a ``point-preserving operation'' $R$, corresponding to either the identity, rotation, inversion, mirror, or the product of mirror and rotation, followed by a translation by some vector $\bm{\tau}$:

\begin{equation}\label{eq:spacegroup}
G = \{(R|\bm{\tau})\}
\end{equation}

where $(R|\bm{\tau})$ means ``Perform $R$, then translate by $\bm{\tau}$''.
Clearly, the set $G$ forms a group, since if both $g_1$ and $g_2 \in G$ leave the crystal structure invariant, so does the product $g_1g_2$, and so $g_1g_2 \in G$.
Thus, $G$ is called the \emph{space group} of the crystal.
In three dimensions, there are $230$ crystallographic space groups, which are tabulated in a number of places, most usefully Wikipedia\cite{wiki:spacegroups}.

For $73$ of these groups, the translation parts of the $\bm{\tau}$s in \cref{eq:spacegroup} are only ever linear combinations of integer multiples of the lattice vectors $\bm{a}$, $\bm{b}$, and $\bm{c}$; these are called \emph{symmorphic} space groups.
The remaining $157$ groups involve translations that are not integer multiples of the lattice vectors; these are one's screw axes and glide planes, and so these groups are called \emph{asymmorphic}.

Importantly, the ``physical properties'' of \cref{thm:neumann} refer to the truly macroscopic properties of the crystal, like it's conductivity, dielectric, or pyroelectric tensors.
Consider, for example, that in \gls{shg}, we are typically studying the sample at optical wavelengths, where the wavelength of light is three or four orders of magnitude larger than the lattice spacing.
Clearly, then, these properties do not care whether the correct symmetry is $(R|\bm{\tau})$ or $(R|\bm{\tau}+\nicefrac{\bm{a}}{2})$.
A more useful group, then, is the \emph{point group} of the crystal

\begin{equation}
P_G = \{R~\suchthat~\exists~\bm{\tau}~\suchthat~(R|\bm{\tau})~\in~G\}
\end{equation}

i.e., the point group is the set of point-preserving operations $R$ for which $R$ appears in $G$, regardless of whether you need to perform a translation with it.
Once can show that $P_G$ is also a group, and thus it is $P_G$ which is involved in Neumann's principle for all intents and purposes\footnote{Of course this breaks down when the wavelength of light is comparable to the lattice spacing; in that case you need to consider the full space group.}.

The last ingredient that we need to understand \cref{thm:neumann} is the concept of what is meant by ``physical propery''.
The idea is that the response of the crystal $J_{i_1i_2\cdots i_n}$ (i.e., the current $J_i$, or the quadrupole moment $Q_{ij}$) is proportional to some field $F_{i_1'i_2'\cdots \i_m'}$ via some tensor $\chi$:
\begin{equation}
J_{i_1i_2\cdots i_n} = \chi_{i_1i_2\cdots i_n i_1' i_2'\cdots i_m'}F_{i_1' i_2' \cdots \i_m'}.
\end{equation}
For example, the conductivity $\sigma_{ij}$ relates a current density $J_i$ to an applied electric field $E_j$:
\begin{equation}
J_i = \sigma_{ij} E_j.
\end{equation}
Likewise, the polarization $P_i$ due to the pyroelectric effect is related to the a temperature difference $\Delta T$ by a tensor $p_i$:
\begin{equation}
P_i = p_i \Delta T.
\end{equation}
The tensors $\sigma_{ij}$, $p_i$, and generally, $\chi_{i_1 i_2 \cdots \i_n i_1' i_2' \cdots i_m'}$ are commonly referred to as \emph{matter tensors}\cite{powell}, to emphasize the fact that they are the only part of the response equations that depend on the material.

We are now ready to restate \cref{thm:neumann}.
