The physics of solids is, to me, one of the most important and fundamental fields of modern science.
This might seem, to some, a bit of a hot take.
After all, by studying condensed matter physics, one learns next to nothing about, say, the formation of the stars and planets, or the origin of the universe.
Nor does one learn about life, death, consciousness, disease, ethics, God, or any other question that perhaps puzzled humanity prior to about five hundred years ago.
Certainly no one would argue that condensed matter physics is quite \emph{useless}, given that nearly every device we interact with in modern life required some condensed matter physicist somewhere along the way to make one brilliant discovery or another -- yet when the human mind starts to wander, and our thoughts turn to the metaphysical, we tend to look up, not down.

In my work I have taken a quite different view.
Condensed matter physics, to me, is ultimately the study of how \textit{truly boring} objects, when brought together in large quantites, \textit{become} interesting, seemingly in spite of themselves.
When electrons are put together in a lattice and allowed to interact slightly with the massive nuclei, at low enough temperatures they pair, the low-energy excitations become gapped, and current can flow for infinite times and with absolutely zero energy loss.
Those same electrons, with some other set of interactions, may instead ionize (the opposite of pairing!) to create an electrically insulating state, whose low-energy excitation spectrum is nevertheless gapless and consisting of charge-neutral spin-$\nicefrac{1}{2}$ particles.
In all such cases, these systems exist in otherwise ordinary-looking rocks, fit in the palm of a hand\footnote{Hopefully, gloved.}, and are more or less indistinguishable from something you might find sticking into the bottom of your shoe.

While such systems may not tell us a lot\footnote{
    This discussion is obviously intentionally reductive.
    In truth there is still quite a bit one can learn about, e.g. the early universe by studying condensed matter physics, see the \citet{kibble_introduction_2008}.
}
about the early universe, considering these and related problems lets us ask deep, fundamental questions about the world we live in -- like, why is this thing a metal, but this thing is an insulator? What do those terms even mean? -- that I don't think we would try to ask otherwise.
To me, focusing our attention on these problems, despite their obviously terrestrial nature, is not a waste of time; rather, I think they remind us that even the most mundane aspects of the human experience involve a level of complexity far beyond what we are capable of understanding absent the pursuit of science.

Throughout the seven years of my Ph.D., I hope to have made a few contributions to this pursuit.
As the title of this work implies, I have mainly focused on the application of ultrafast techniques to the study of correlated quantum materials, which I loosely define as those materials in which the interaction between particles is large enough so as to compete with the kinetic energy of those particles.
It is in these materials that I think lies the true frontier of condensed matter physics; here, much of our basic intuition about non- or weakly-interacting theory fails, and more complicated notions of phase competition, phase separation, disorder, pairing, coherence, etc. are needed to property describe the relevant physics.

In my own view, and in the view of many scientists in this field\citep{alexandradinata_future_2022}, the main question for strongly correlated physics amounts to: ``Given a correlated system with some defined combination of different interaction strengths, is there a general theory which allows us to predict the phase diagram of this system \textit{a priori}?''
Related of course are questions about the origins of high-$T_c$ superconductivity, strange metallicity, quantum spin liquids, and other exotic phases that we find emerging from strongly interacting systems.
Since such a theory does not currently exist, at least with the level of predictive power that I think most would find satisfactory, new advances in this field typically come directly from experiment.
Ultrafast optics plays a special role in this regard, for reasons that I will explain in \cref{ch:ch1}.

Progress thus happens in this field somewhat unsystematically, with small pieces of the puzzle added at random, but not infrequenct, intervals.
Usually it is either new techniques or new materials that are the driving force here.
To this end, I have tried to pursue both directions in my Ph.D.
Appearing also in \cref{ch:ch1} is thus a description of the materials I studied the most during my thesis, two of them, \ce{CuBr2} and \ce{CaMn2Bi2} I consider criminally understudied.
On the technique side, almost all of the work presented in this thesis was done using \gls{trshg}, a relatively new, nonlinear optical technique which, at the most basic level, probes the point group assumed by the charge distribution function $\rho(\bm{x})$ at any given point in time.
\Gls{shg} and \gls{trshg} are tricky techniques, with many pitfalls both practically and theoretically; \cref{ch:ch2,ch:ch3} are thus devoted to what I hope is a useful, if not fully compehensive, description of the technique.
My hope is that these sections are useful not only for the new student trying to build their own setup or analyze their own \gls{shg} data, but also for people for whom \gls{shg} is not a focus but nevertheless want to learn about it in slightly more detail than one would get from a typical paper or review article.
Some aspects of \cref{ch:ch3} are devoted to work that we did developing a new way to control the polarization of the light in a \gls{trshg} experiment using stepper motors.

What follows, then, is a description of the three main research works I contributed during my Ph.D..
The first, which I describe in \cref{ch:ch4}, involves work that I did during my second and third years on \tastwo, a very interesting \gls{cdw} material that, among other things, undergoes a mirror symmetry breaking \gls{cdw} transition at $350$ \si{K} that shows up in the \gls{shg} as a sudden distortion of the flower pattern at that temperature.
Since this transition breaks mirror symmetry, two energetically degenerate domains should be present, corresponding to two opposite planar chiralities; in this work, we showed that \gls{shg} could differentiate between these two domains (i.e. the flower pattern in either domain looks different).

The second and third works, which I describe in \cref{ch:ch5,ch:ch6}, in contrast to the \tastwo work, both involve taking the system out of equilibrium to study the dynamics.
In \ce{CaMn2Bi2} (\cref{ch:ch5}), we discovered that photoexcitation causes the \gls{afm} order in that compound to reorient (relative to equilibrium) to a metastable state which is impossible to reach from the equilibrium state thermodynamically.
Light is thus used to \textit{control} the magnetic order in this material.

In \ce{CuBr2} (\cref{ch:ch6}), light is not used to control the order parameter like in \ce{CaMn2Bi2}, but it does excite coherent oscillations of the collective modes of the multiferroic order (electromagnons), whose frequency, amplitude, damping, etc. may be probed in \gls{trshg} as a function of temperature -- a methodology referred to as ultrafast \textit{spectroscopy}.
In doing so, we found that one of these collective modes is actually quite special, as it is in fact the analogue of the Higgs mode of particle physics in the context of a multiferroic material.

I conclude with various remarks in \cref{ch:ch7}, as well as an appendix, in which I enumerate briefly all of the null-result experiments I performed during my Ph.D., in the hopes that future scientists don't have to waste time on what we already know are fruitless pursuits.
If you have any questions about this or any other section of this thesis, please do not hesitate to reach out via email.
