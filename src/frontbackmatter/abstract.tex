\thispagestyle{empty}
\calccentering{\unitlength}
\begin{adjustwidth*}{\unitlength}{-\unitlength}
\begin{center}
{\Large\bfseries Ultrafast spectroscopy and control of correlated quantum materials}\\[0.5\baselineskip]
{By}\\[0.5\baselineskip]
{\large Bryan T. Fichera\\[0.5\baselineskip]}
Submitted to the Department of Physics\\ on \defenddate in Partial Fulfillment of the Requirements for the Degree of Doctor of Philosophy\\[\baselineskip]
\end{center}
\noindent {\large\textbf{Abstract}}\\[\baselineskip]
In this thesis, I describe research completed during my Ph.D. on correlated condensed matter systems using ultrafast optics.
I begin with a broad overview of this field, focusing specifically on the essential physics involved in ultrafast processes and how that physics may be utilized, in the sense of either spectroscopy or control, to understand correlated systems.
I then give a pedagogical introduction to \glsfmtlong{shg}, both in theory and in practice, before describing results from four projects I completed in my Ph.D.---(i) a technical project concerned with automating polarization rotation in \glsfmtlong{shg}, (ii), a demonstration that \glsfmtlong{shg} may be used to differentiate \glsfmtlong{cdw} domains with opposite planar chirality, (iii) our discovery of an ultrafast reorientation transition in the antiferromagnetic semiconductor \ce{CaMn2Bi2}, and (iv) \glsfmtlong{shg} evidence for \ahiggs-mode electromagnon in \ce{CuBr2}.
I conclude by reflecting on the progress achieved in correlated electron physics as a result of this work, and by giving my own perspective on the future of this field.\\[\baselineskip]
\noindent Thesis supervisor: Nuh Gedik\\[\baselineskip]
\noindent Title: Donner Professor of Physics
\end{adjustwidth*}
