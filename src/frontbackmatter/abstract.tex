In this thesis, I describe research on correlated condensed matter systems using ultrafast optics which I completed during my Ph.D.
I begin with an introduction to the field of ultrafast optics in correlated systems, in which I compare ultrafast spectroscopy to ultrafast control and discuss the interplay between these two related fields.
Then, I give a pedagogical introduction to \glsfmtlong{shg}, both in theory and in practice.
I proceed to describe four research works from my Ph.D.---(i) progress in automating polarization rotation in \glsfmtlong{shg}, (ii), probing broken inversion symmetry with \glsfmtlong{shg}, (iii) an ultrafast reorientation transition in the antiferromagnetic semiconductor \ce{CaMn2Bi2}, and (iv) observation of \ahiggs-mode electromagnon in \ce{CuBr2}.
I conclude with a few remarks on the progress achieved in this work and a brief outlook on future research in this field.
